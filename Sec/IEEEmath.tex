\newpage
\newcommand{\TO}{$\quad\to\quad$}
%------------------------------------------------------------------------
\section{IEEE Style: \texttt{Mathematics} environment for IEEEtran.cls}
%------------------------------------------------------------------------
\raggedright{\uwave {\bfseries{Some Examples:}}}
%
\subsection{In-Line Equations}
\begin{center} \verb!$\frac{1}{x^2}$! \TO $\frac{1}{x^2}$\par
This is \verb!$\dfrac{1}{x^2}$! that \TO $\dfrac{1}{x^2}$ 
\end{center}

 
\subsection{Display Equations}
\begin{center} \verb!$$\frac{dx}{dt}$$! \TO  \end{center}
$$\frac{1}{x^2}$$

\begin{center} \verb!$$\dfrac{1}{x^2}$$! \TO  \end{center}
$$\dfrac{1}{x^2}$$
Alternatively, it is
\begin{center} \verb!\[\dfrac{1}{x^2}\]! \TO  \end{center}
\[\dfrac{1}{x^2}\]
%
%
%------------------------------------------------------------------------
\subsection{Enumerated Equations}
%------------------------------------------------------------------------
Equations are created using the traditional equation environment as shown below in \eqref{eq:ieee1} (\verb!\eqref{eq:ieee1}!):
%
\begin{verbatim}
\begin{equation}  \label{eq:ieee1}
x \:=\: \sum\limits_{i=0}^{z} 2^{i}Q\,.
\end{equation}
\end{verbatim}
%
which yields
%
\begin{equation}  \label{eq:ieee1}
x\:=\: \sum\limits_{i=0}^{z} 2^{i}Q\,.
\end{equation}
%
%
%------------------------------------------------------------------------
\subsection{Non-Enumerated Equations}
%------------------------------------------------------------------------
%
\begin{verbatim}
\begin{equation*}  
\mathbf{H}(s)\:=\:\left(G\:+\:sC \right)^{-1}B
\end{equation*}
\end{verbatim}
%
i.e. 
%
\begin{equation*}  
	\mathbf{H}(s)\:=\:\left(G\:+\:sC \right)^{-1}B
\end{equation*}
%
%
%------------------------------------------------------------------------
\subsection{IEEE: Long Equation in Multi-line Format}
%------------------------------------------------------------------------
If an equation is too big to be written in one line We also can split it in two or more lines as shown below. 
%
%
\subsubsection{Alt-1\_Using \texttt{multline} environment:}
%
%
\begin{verbatim}
\begin{multline} \label{eq:1001}
	H\:=\:\sum\limits_{j}A+B\:=\: \\
	\sum\limits_{j}C+D+ \ldots
\end{multline}
\end{verbatim}
%
%
\begin{center}
\begin{minipage}{0,4\textwidth}
\begin{multline} \label{eq:1001}
	H\:=\:\sum\limits_{j}A+B\:=\: \\
	\sum\limits_{j}C+D+ \ldots
\end{multline}
\end{minipage}
\end{center}
%
%
\subsubsection{Alt-2\_Using \texttt{spilt} Environment:}
This is recommended for the typical IEEE style!
%
%
\begin{verbatim}
\begin{equation} 
\begin{split}
H_c &=  \sum\limits_{l_1+\dots+ l_p=l}\prod\limits^p_{i=1} \binom{n_i}{l _i} \\
&\quad \times [(n-l )]^n \times \sum\limits^p_{j=1}(n_i)^2
\end{split}
\end{equation}
\end{verbatim}
%
i.e.
%
\begin{equation} 
\begin{split}
H_c &=  \sum\limits_{l_1+\dots+ l_p=l}\prod\limits^p_{i=1} \binom{n_i}{l _i} \\
&\quad \times [(n-l )]^n \times \sum\limits^p_{j=1}(n_i)^2
\end{split}
\end{equation}
%
%
%------------------------------------------------------------------------
\subsection{IEEE: Multi-line  Equations}
%------------------------------------------------------------------------
%
%------------------------------------------------------------------------
\subsubsection{\texttt{\Large eqnarray} Environment}
%------------------------------------------------------------------------
To restore IEEEtran's ability to automatically break within multiline equations, load \texttt{amsmath} in preamble like:
%
\begin{verbatim}
\usepackage{amsmath}
\interdisplaylinepenalty=2500
\end{verbatim}
%
%
\raggedright{\uwave {\bfseries{Example 1:}}}
%
%
\begin{verbatim}
\setlength{\arraycolsep}{0.0em}
\begin{eqnarray}
Z&{}={}&x_1 + x_2 + x_3 + x_4 + x_5 + x_6\nonumber\\
 &     &+a +b\\
 &     &+C +d +e  \nonumber\\
 &  =  &+f +g +h +m +n
\end{eqnarray}
\setlength{\arraycolsep}{5pt}
\end{verbatim}
%
%
gives:
%
%
\setlength{\arraycolsep}{0.0em}
\begin{eqnarray}
Z&{}={}&x_1 + x_2 + x_3 + x_4 + x_5 + x_6\nonumber\\
&     &+a +b\\
&     &+C +d +e  \nonumber\\
&  =  &+f +g +h +m +n
\end{eqnarray}
\setlength{\arraycolsep}{5pt}
%
%
\raggedright{\uwave {\bfseries{Example 2:}}}\\
%
This example is from ``IEEEtran\_HOWTO.pdf''
%
%
\begin{verbatim}
\setlength{\arraycolsep}{0.0em}
\begin{eqnarray}
Z&{}={}&x_1 + x_2 + x_3 + x_4 + x_5 + x_6\nonumber\\
&&+a + b\\
&&+{}a + b\\
&&{}+a + b\\
&&{+}\:a + b
\end{eqnarray}
\setlength{\arraycolsep}{5pt}
\end{verbatim}
%
%
Results in
\setlength{\arraycolsep}{0.0em}
\begin{eqnarray}
Z&{}={}&x_1 + x_2 + x_3 + x_4 + x_5 + x_6\nonumber\\
&&+a + b\\
&&+{}a + b\\
&&{}+a + b\\
&&{+}\:a + b
\end{eqnarray}
\setlength{\arraycolsep}{5pt}
%
%
Perhaps ``eqnarray environment'' is the most convenient and popular way to produce multiline equations. However, \texttt{eqnarray} has several serious shortcomings (see IEEEtran\_HOWTO.pdf).
%
%
%------------------------------------------------------------------------
\subsubsection{\texttt{\Large flalign} for Putting Text in the Equation}
%------------------------------------------------------------------------
\verb|flalign| environment is used to put a text in the same line with equation and flushing it to left as shown below.
%
\\ Note: This can be useful to save some space in conference paper but in journals they do not generally use this.
%
%
%------------------------------------------------------------------------
\raggedright{\uwave {\bfseries{Example 1:}}}\\
%------------------------------------------------------------------------
\begin{flalign} 
	&\text{It is}& PF(\alpha) &= A+B  &     \label{h-1a} \\
	&\text{Where}& UC(\alpha) &= x    &     \nonumber    \\
	&            &            &= W    &     \label{h-1b}
\end{flalign}
%
by
%
\begin{verbatim}
\begin{flalign} 
&\text{It is}& PF(\alpha) &= A+B  &     \label{h-2a} \\
&\text{where}& UC(\alpha) &= x    &     \nonumber    \\
&            &            &= W    &     \label{h-2b}
\end{flalign}
\end{verbatim}
%
%
%------------------------------------------------------------------------
\raggedright{\uwave {\bfseries{Example 2:}}}\\
%------------------------------------------------------------------------
\begin{flalign} 
	&\text{L.H.S:}& PF(\alpha) &= A+B  &     \label{h-3a} \\
	&\text{R.H.S:}& UC(\alpha) &= x    &     \nonumber    \\
	&&                         &= W    &     \label{h-3b}
\end{flalign}
%
by
%
\begin{verbatim}
\begin{flalign} 
&\text{L.H.S:}& PF(\alpha) &= A+B  &     \label{h-2a} \\
&\text{R.H.S:}& UC(\alpha) &= x    &     \nonumber    \\
&&                         &= W    &     \label{h-2b}
\end{flalign}
\end{verbatim}
%
%
%------------------------------------------------------------------------
\subsubsection{\texttt{\Large align} Environment}
%------------------------------------------------------------------------
%
%------------------------------------------------------------------------
\raggedright{\uwave {\bfseries{Example 1:}}}\\
%------------------------------------------------------------------------
\begin{verbatim}
\begin{align} 
A\:&=\:B  \label{L1-a}\\
C\:&=\:D  \label{L1-b}
\end{align} 
\end{verbatim}
%
%
\begin{align} 
A\:&=\:B  \label{L1-a}\\
C\:&=\:D  \label{L1-b}
\end{align}
%
%
%------------------------------------------------------------------------
\raggedright{\uwave {\bfseries{Example 2:}}}\\
%------------------------------------------------------------------------
%
\begin{verbatim}
\begin{align} \label{eq:x1}
A\:=\:1 && 
B\:=\:2 && 
C\:=\:3 && 
D\:=\:4 &&
D\:=\:5
\end{align}
\end{verbatim}
%
\begin{align} \label{eq:x1}
A\:=\:1 && 
B\:=\:2 && 
C\:=\:3 && 
D\:=\:4 &&
D\:=\:5
\end{align}
%
\begin{verbatim}
\begin{align}    
A\:+\:B &\:=\:C  \nonumber\\ 
&\:=\:D          \label{L1}
\end{align}
\end{verbatim}
%
%
\begin{align}   
A\:+\:B &\:=\:C  \nonumber\\ 
&\:=\:D          \label{L1}
\end{align}
%
%
Equivalently this can be done as:
%
\begin{verbatim}
\begin{align}    \label{L1} 
A\:+\:B &\:=\:C  \nonumber\\ 
&\:=\:D          
\end{align}
\end{verbatim}
%
%
\begin{align}    \label{L1} 
A\:+\:B &\:=\:C  \nonumber\\ 
&\:=\:D          
\end{align}
%
%

%------------------------------------------------------------------------
\raggedright{\uwave {\bfseries{Example 3:}}}\\
%------------------------------------------------------------------------
\begin{verbatim}
\begin{align}
x   &=y    & X  &=Y    & a &=b+c\\
x   &=y    & X  &=Y    & a &=b\\
x+x &=y+y  & X+X&=Y+Y  & ab&=cb
\end{align}
\end{verbatim}
%
%
\begin{align}
	x   &=y    & X  &=Y    & a &=b+c\\
	x   &=y    & X  &=Y    & a &=b\\
	x+x &=y+y  & X+X&=Y+Y  & ab&=cb
\end{align}
%
%
%------------------------------------------------------------------------
\subsubsection{\texttt{\Large alignat}~Environment}
%------------------------------------------------------------------------
%
%------------------------------------------------------------------------
\raggedright{\uwave {\bfseries{Example 1:}}}\\
%------------------------------------------------------------------------
%
\begin{verbatim}
\begin{alignat}{2}\label{t7}
\tilde{C}\, &= \,Q^TCQ, \quad \qquad    &\tilde{G}\, &= \,Q^TGQ, \nonumber\\ 
\tilde{B}\, &= \,Q^TB,   \;\text{and}    &\tilde{L}\,&= \,LQ\,. 
\end{alignat}
\end{verbatim}
%
%
\begin{alignat}{2}\label{t7}
\tilde{C}\, &= \,Q^TCQ, \quad \qquad    &\tilde{G}\, &= \,Q^TGQ, \nonumber\\ 
\tilde{B}\, &= \,Q^TB,   \;\text{and}    &\tilde{L}\,&= \,LQ\,. 
\end{alignat}
%
%
%------------------------------------------------------------------------
\raggedright{\uwave {\bfseries{Example 2:}}}\\
%------------------------------------------------------------------------
%
\begin{verbatim}
\begin{alignat}{3}\label{eq:t11}
	\mathat{C}\, &\trieq\,\mat{V}\T\mat{C}\mat{V}\: \in \mathbb{R}^{m\times m} 
	\quad \quad    
	&\mathat{B}\, &\trieq\,\mat{V}\T\mat{B} \in  \mathbb{R}^{m\times n_{in}} 
	\nonumber\\ 
	\mathat{G}\,&\trieq\,\mat{V}\T\mat{G}\mat{V}\: \in \mathbb{R}^{m\times m} 
	&\mathat{L}\, &\trieq\,\mat{L}\mat{V}\;\;   \in \mathbb{R}^{n_{out}\times m}
\end{alignat}
\end{verbatim}
%
%
\begin{alignat}{3}\label{eq:t11}
	\mathat{C}\, &\trieq\,\mat{V}\T\mat{C}\mat{V}\: \in \mathbb{R}^{m\times m} 
	\quad \quad    
	&\mathat{B}\, &\trieq\,\mat{V}\T\mat{B} \in  \mathbb{R}^{m\times n_{in}}    \nonumber\\ 
	\mathat{G}\,&\trieq\,\mat{V}\T\mat{G}\mat{V}\: \in \mathbb{R}^{m\times m} 
	&\mathat{L}\, &\trieq\,\mat{L}\mat{V}\;\;   \in \mathbb{R}^{n_{out}\times m}
\end{alignat}
%
%
%------------------------------------------------------------------------
\subsubsection{\texttt{\Large aligned}~Environment}
%------------------------------------------------------------------------
%
\begin{verbatim}
\begin{equation}
\left.\begin{aligned}
B'&=-\partial \times E,\\
E'&=\partial \times B - 4\pi j,
\end{aligned}
\right\}
\qquad \text{Maxwell's equations}
\end{equation}
\end{verbatim}
%
%
\begin{equation}
	\left.\begin{aligned}
		B'&=-\partial \times E,\\
		E'&=\partial \times B - 4\pi j,
	\end{aligned}
	\right\}
	\qquad \text{Maxwell's equations}
\end{equation}
%
%
%------------------------------------------------------------------------
\subsubsection{\texttt{\Large subequations} Environment}
%------------------------------------------------------------------------
This is working \texttt{IEEEtran.sty} very well!
%
%
%------------------------------------------------------------------------
\par \raggedright{\uwave {\bfseries{Example 2:}}}\\
%------------------------------------------------------------------------
Using \texttt{subequations} \& \texttt{\textbf{equation}} environments together:
%
\begin{verbatim}
\begin{subequations}
\begin{equation}
\rho(r)=\frac{\sigma^2}{2\pi G r_c^2}
\frac{1}{1+\left(\frac{r}{r_c}\right)^2}
\label{t9}
\end{equation}
\begin{equation}
M_{PID}(R) = \frac{\sigma^2}{G}
2\left[ R - r_c \arctan\left(\frac{R}{r_c}\right)\right]
\label{t10}
\end{equation}
\end{subequations}
\end{verbatim}
%
%
\begin{subequations}
	\begin{equation}
	\rho(r)=\frac{\sigma^2}{2\pi G r_c^2}
	\frac{1}{1+\left(\frac{r}{r_c}\right)^2}
	\label{t11}
	\end{equation}
	\begin{equation}
	M_{PID}(R) = \frac{\sigma^2}{G}
	2\left[ R - r_c \arctan\left(\frac{R}{r_c}\right)\right]
	\label{t12}
	\end{equation}
\end{subequations}
%
%
%------------------------------------------------------------------------
\raggedright{\uwave {\bfseries{Example 2:}}}\\
%------------------------------------------------------------------------
Using \texttt{subequations} \& \textbf{\texttt{align}} environments together:
%
%
\begin{verbatim}
\begin{subequations} \label{eq:PEEC_form1}
\begin{align}
\brk{\mat{G}(\bs{\xi})\:+\: s \mat{C}(\bs{\xi})}\mat X(s,\xi) \:&=\:
\mat{B}\mat{I}_p(s) \label{eq:PEEC_form1_a}\\
\mat{V}_p(s,\bs{\xi})\:&=\:
\mat{L}\mat{X}(s,\:\bs{\xi}) \label{eq:PEEC_form1_b}
\end{align}
\end{subequations}
\end{verbatim}
%
%
\begin{subequations} \label{eq:PEEC_form1}
	\begin{align}
		\brk{\mat{G}(\bs{\xi})\:+\: s \mat{C}(\bs{\xi})}\mat X(s,\xi)\: &= \: \mat{B}\mat{I}_p(s) \label{eq:PEEC_form1_a}\\
		\mat{V}_p(s,\bs{\xi})\:&=\:\mat{L}\mat{X}(s,\:\bs{\xi}) \label{eq:PEEC_form1_b}
	\end{align}
\end{subequations}
%
%
%------------------------------------------------------------------------
\raggedright{\uwave {\bfseries{Example 3:}}}\\
%------------------------------------------------------------------------
Using \texttt{subequations} environment with the \textbf{\texttt{align}} environment:
%
%
\begin{verbatim}
\begin{subequations} \label{eq:124}
\begin{align}   
s\mat{C}(s)\:&=\:-\mat{G}(s)\label{eq:124_a}\\
\mat{I}\:&=\:\mat{B}        \label{eq:124_b}\\
\mat{C}\:&=\:\mat{C}        \label{eq:124c}\\
\mat{G}\:&=\:\mat{G}        \label{eq:124_d}
\end{align}
\end{subequations}
\end{verbatim}
%
%
\begin{subequations} \label{eq:124}
	\begin{align}   
		s\mat{C}(s)\:&=\:-\mat{G}(s)\label{eq:124_a}\\
		\mat{I}\:&=\:\mat{B}        \label{eq:124_b}\\
		\mat{C}\:&=\:\mat{C}        \label{eq:124_c}\\
		\mat{G}\:&=\:\mat{G}        \label{eq:124_d}
	\end{align}
\end{subequations}
%
All the equations are referred to as \verb!\eqref{eq:124}! \eqref{eq:124} and then each one separately as \verb!\eqref{eq:124_a}! \eqref{eq:124_a} ,...,\verb!\eqref{eq:124_d}! \eqref{eq:124_d}.
%
%------------------------------------------------------------------------
\par \raggedright{\uwave {\bfseries{Example 4:}}}\\
%------------------------------------------------------------------------
Using \texttt{subequations} environment with the \textbf{\texttt{align}} environment:
%
\begin{subequations} \label{eq:red_gad1}
	\begin{align}
		s\mathat{C}(s) \mathat{X}(s)\:+\:\mathat{G}\mathat{X}(s)\:&=\:\mathat{B}\mat{U}(s) \label{eq:red_gad1_a}\\
		\mat{I}_p(s)\:&=\:\mathat{B}\T\mathat{X}(s) \label{eq:red_gad1_b}
	\end{align}
	\text{where}	
	\begin{align}
		\mathat{G}&\trieq\mat{Q}\T\mat{G}\mat{Q}&
		\mathat{C}&\trieq\mat{Q}\T\mat{C}\mat{Q}&
		\mathat{B}&\trieq\mat{Q}\T\mat{B}&
	\end{align}
\end{subequations}
%
%
\begin{verbatim}
\begin{subequations} \label{eq:red_gad1}
\begin{align}
s\mathat{C}(s) \mathat{X}(s)\:+\:\mathat{G}\mathat{X}(s)\:&=
\:\mathat{B}\mat{U}(s) \label{eq:red_gad1_a}\\
\mat{I}_p(s)\:&=\:\mathat{B}\T\mathat{X}(s) \label{eq:red_gad1_b}
\end{align}
\text{where}	
\begin{align}
\mathat{G}&\trieq\mat{Q}\T\mat{G}\mat{Q}&
\mathat{C}&\trieq\mat{Q}\T\mat{C}\mat{Q}&
\mathat{B}&\trieq\mat{Q}\T\mat{B}&
\end{align}
\end{subequations}
\end{verbatim}
%
%------------------------------------------------------------------------
\subsubsection{\texttt{\Large case} structure: for multi-line equations}
%------------------------------------------------------------------------
This (\texttt{case} structure does not accept \texttt{\label{}})
%
%
\begin{verbatim}
\begin{equation*}
\left|x\right|= 
\begin{cases} 
XX & \text{if} X=0,     \\
YY & \text{if} YY\le 0  
\end{cases}
\end{verbatim}
%
%
\begin{equation*}
\left|x\right|= 
\begin{cases} 
XX & \text{if} X=0,     \\
YY & \text{if} YY\le 0  
\end{cases}
\end{equation*}
%
%
%------------------------------------------------------------------------
\subsubsection{\texttt{\Large subnumcases} Environment}
%------------------------------------------------------------------------
%
With a name at behind of the curly bracket, here, e.g., it is $\Psi$:
%
\begin{verbatim}
\begin{subnumcases}{\label{t3} \Psi\,:\;}
C\frac{d}{{dt}}x(t)\; + \;Gx(t)\; = \;B\mathbf{u}(t)  \label{t3_a}\\
\mathbf{i}(t)\; = \;\mathbf{L}x(t)\,,  \label{t3_b}
\end{subnumcases}
\end{verbatim}
%
%
\begin{subnumcases}{\label{t4} \Psi\,:\;}
C\frac{d}{{dt}}x(t)\; + \;Gx(t)\; = \;B\mathbf{u}(t)  \label{t4_a}\\
\mathbf{i}(t)\; = \;\mathbf{L}x(t)\,,  \label{t4_b}
\end{subnumcases}
%
%
\vskip 8pt
And if without a name at behind of the curly bracket is intended:
%
\begin{verbatim}
\begin{subnumcases}{\label{t3} {}} %This {} should be used!
C\frac{d}{{dt}}x(t)\; + \;Gx(t)\; = \;B\mathbf{u}(t)  \label{t3_a}\\
\mathbf{i}(t)\; = \;\mathbf{L}x(t)\,,  \label{t3_b}
\end{subnumcases}
\end{verbatim}
%
%
\begin{subnumcases}{\label{t40} {}}
C\frac{d}{{dt}}x(t)\; + \;Gx(t)\; = \;B\mathbf{u}(t)  \label{t40_a}\\
\mathbf{i}(t)\; = \;\mathbf{L}x(t)\,,  \label{t40_b}
\end{subnumcases}
%
%
Note that in \texttt{subnumcases} environment the numbers for equations includes letters 'a' and 'b', hence the name, 'subnum'; e.g. (1a) and (1b).  
%
%
%------------------------------------------------------------------------
\subsection{\texttt{\Large numcases} Environment}
%------------------------------------------------------------------------
%
Unlike \texttt{subnumcases} the numbers for equations does not include; e.g. it is (1) and (2).
%
%
\begin{verbatim}
\begin{numcases}{\label{t101} {\Sigma:}}
C\frac{d}{{dt}}x(t)\; + \;Gx(t)\; = \;B\mathbf{u}(t)  \label{t101_a}\\
\mathbf{i}(t)\; = \;\mathbf{L}x(t)\,,  \label{t101_b}
\end{numcases}
\end{verbatim}
%
%
\begin{numcases}{\label{t102} {\Sigma:}}
C\frac{d}{{dt}}x(t)\; + \;Gx(t)\; = \;B\mathbf{u}(t)  \label{t102_a}\\
\mathbf{i}(t)\; = \;\mathbf{L}x(t)\,,  \label{t102_b}
\end{numcases}
%
%

%------------------------------------------------------------------------
\subsection{\texttt{\Large gather}: Equation groups without alignment}
%------------------------------------------------------------------------
%
%------------------------------------------------------------------------
\par \raggedright{\uwave {\bfseries{Example 1:}}}\\
%------------------------------------------------------------------------
%
\begin{gather} \label{t16}
	H_c = \\
	\begin{split}
		&= a^2\\
		&= b^c\\
	\end{split}
	\\
	= H_d
\end{gather}
%
by
%
\begin{verbatim}
\begin{gather} \label{t15}
H_c = \\
\begin{split}
&= a^2\\
&= b^c\\
\end{split} \\
= H_d
\end{gather}
%
\end{verbatim}
%
%
%\begin{center}{\rule{10cm}{0.5pt}}\end{center}
%
%------------------------------------------------------------------------
\par \raggedright{\uwave {\bfseries{Example 2:}}}\\
%------------------------------------------------------------------------
%
%
\begin{gather*}
	a_0=\frac{1}{\pi}\int\limits_{-\pi}^{\pi}f(x)\,\mathrm{d}x\\[6pt]
	\begin{split}
		a_n=\frac{1}{\pi}\int\limits_{-\pi}^{\pi}f(x)\cos nx\,\mathrm{d}x=\\
		=\frac{1}{\pi}\int\limits_{-\pi}^{\pi}x^2\cos nx\,\mathrm{d}x
	\end{split}\\[6pt]
	\begin{split}
		b_n=\frac{1}{\pi}\int\limits_{-\pi}^{\pi}f(x)\sin nx\,\mathrm{d}x=\\
		=\frac{1}{\pi}\int\limits_{-\pi}^{\pi}x^2\sin nx\,\mathrm{d}x
	\end{split}\\[6pt]
\end{gather*}
%
by
%
\begin{verbatim}
\begin{gather*}
a_0=\frac{1}{\pi}\int\limits_{-\pi}^{\pi}f(x)\,\mathrm{d}x\\[6pt]
	\begin{split}
	a_n=\frac{1}{\pi}\int\limits_{-\pi}^{\pi}f(x)\cos nx\,\mathrm{d}x=\\
	=\frac{1}{\pi}\int\limits_{-\pi}^{\pi}x^2\cos nx\,\mathrm{d}x
	\end{split}\\[6pt]
	\begin{split}
	b_n=\frac{1}{\pi}\int\limits_{-\pi}^{\pi}f(x)\sin nx\,\mathrm{d}x=\\
	=\frac{1}{\pi}\int\limits_{-\pi}^{\pi}x^2\sin nx\,\mathrm{d}x
	\end{split}\\[6pt]
\end{gather*}
\end{verbatim}
%
%
%------------------------------------------------------------------------
\par \raggedright{\uwave {\bfseries{Example 3:}}}\\
%------------------------------------------------------------------------
%
%
\begin{gather}
	A = \{\, x \mid x \in X_{i}, \text{ for some } i \in I \,\}          \nonumber\\
	A = \{\, x \mid \text{for $x$ large} \,\}                            \nonumber\\
	a_{\text{left}} + 2 = a_{\text{right}}                               \nonumber\\
	\sum_{i = 1}^{ \left[ \frac{n}{2} \right] }
	\binom{ x_{i, i + 1}^{i^{2}} }
	{ \left[ \frac{i + 3}{3} \right] }
	\frac{ \sqrt{ \mu(i)^{ \frac{3}{2}} (i^{2} - 1) } }
	{\sqrt[3]{\rho(i)-2} + \sqrt[3]{\rho(i) - 1}}                       \nonumber\\   
	\left[ \sum_i a_i \right]^{1/p} \quad
	\mm{\sum_i a_i}^{1/p}                                    \nonumber\\         
\end{gather}
%
by
%
\begin{verbatim}
\begin{gather}
A = \{\, x \mid x \in X_{i}, \text{ for some } i \in I \,\}          \nonumber\\
A = \{\, x \mid \text{for $x$ large} \,\}                            \nonumber\\
a_{\text{left}} + 2 = a_{\text{right}}                               \nonumber\\
\sum_{i = 1}^{ \left[ \frac{n}{2} \right] }
\binom{ x_{i, i + 1}^{i^{2}} }
{ \left[ \frac{i + 3}{3} \right] }
\frac{ \sqrt{ \mu(i)^{ \frac{3}{2}} (i^{2} - 1) } }
{\sqrt[3]{\rho(i)-2} + \sqrt[3]{\rho(i) - 1}}                       \nonumber\\   
\left[ \sum_i a_i \right]^{1/p} \quad
\mm{\sum_i a_i}^{1/p}                                      \nonumber\\         
\end{gather}
\end{verbatim}
%
%
%==========================================================================
\begin{center}\deco{10pt}{44444444444}\end{center}
%==========================================================================

