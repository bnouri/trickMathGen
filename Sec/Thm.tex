\section{Theorem Environment}
\texttt{amsthm} provides:  Theorem environments, \verb@\begin{proof}@ and a  \verb@\qedhere@ which produces \emph{tombstones} for the ends of proofs.
 
 The following code will further define theorem, lemma, proposition, corollary, proof, definition, example and remark environments, together with a control sequence \verb@\qed@ which produces `tombstones' for the ends of proofs!

\subsection{Types of theorems and their formats}
%
%
\textbf{\emph{Plain Types:}}\\
\emph{Theorem, Lemma, Corollary, Proposition, Conjecture,Criterion, Algorithm}\\
{$\bullet$~~}Makes theorem's body italic \\

\vspace{12pt}
\textbf{\emph{Definition Types:}}\\
\emph{definition, Condition, Problem, Example}\\
{$\bullet$~~}Makes theorem's body roman\\

\vspace{12pt}
\textbf{\emph{Remarks Types:}}\\
\emph{Remark, Question, Claim, Note, Fact, Case, Notation, Summary, Acknowledgment, Conclusion}\\
{$\bullet$~~}Makes theorem's body non-italic\\

\subsection{Definitions}
\begin{description}
%%========================================================================
%
\item[Theorem:]{test} 
\item[Lemma:]   
\item[Corollary:]  
\item[Proposition:]  
\item[Conjecture:]
\item[Criterion:]
\item[Algorithm:]
\item[Remark:]
\item[Question:]
\item[Claim:]
\item[Note:]
\item[Fact:]
\item[Case:]
\item[Notation:]
\item[Summary:]
\item[Acknowledgment:]
\item[Conclusion:]
%=============================================================================
\end{description}
%
%
%------------------------------------------------------------------------
\subsection{\texttt{\Large proof} Environment}
%------------------------------------------------------------------------
%
%------------------------------------------------------------------------
\par \raggedright{\uwave {\bfseries{Example 1:}}}\\
%------------------------------------------------------------------------
%
\begin{proof}
Here it is my proof	
	\begin{equation}
	a^2 + b^2 = c^2
	\end{equation}
\end{proof}
%
by
%
\begin{verbatim}
\begin{proof}
Here it is my proof	
\begin{equation}
a^2 + b^2 = c^2
\end{equation}
\end{proof}
\end{verbatim}
%
%
Please compare it with the following which is using \verb!\qedhere!.
%
\begin{proof}
	Here is my proof
	\begin{equation}
		a^2 + b^2 = c^2 \qedhere
	\end{equation}
\end{proof}
%
by
%
\begin{verbatim}
\begin{proof}
Here is my proof
\begin{equation}
a^2 + b^2 = c^2 \qedhere
\end{equation}
\end{proof}
\end{verbatim}
%
%
Also, for the \ding{110} at the end of the proof, without starting a newline, we can use \verb!\QED! to obtain: \QED
%
%
%------------------------------------------------------------------------
\par \raggedright{\uwave {\bfseries{Example 2:}}}\\
%------------------------------------------------------------------------
The method above does not work with the deprecated environment \texttt{eqnarray*}. Here is a workaround:
%
%
\begin{proof}
Here is my proof:
\begin{eqnarray*}
a^2 + b^2 = c^2
\end{eqnarray*}
\vspace{-1.3cm}\[\qedhere\]
\end{proof}
%
%
\begin{proof}
proof goes here!
\begin{equation}
G(t)=L\gamma!\,t^{-\gamma}+t^{-\delta}\eta(t) %\qedhere
\end{equation}
\end{proof}
%
%
%------------------------------------------------------------------------
\par \raggedright{\uwave {\bfseries{Example 3:}}}\\
%------------------------------------------------------------------------
By using ``\verb@\SOL this is my solution.@'' we obtain a result as shown below.\\[6pt]
\SOL this is my solution.\\

%
%===============================================================================
%Theorems:
%
%
\begin{theorem}
Try this1
\begin{equation}
G(t)=L\gamma!\,t^{-\gamma}+t^{-\delta}\eta(t) \qedhere
\end{equation}
\end{theorem}
%
%
\begin{question} \setcounter{equation}{0}
Try this2
\begin{equation}
G(t)=L\gamma!\,t^{-\gamma}+t^{-\delta}\eta(t) \qedhere
\end{equation}
\end{question}
%
%
\begin{question} \setcounter{equation}{0}
Try this3
\begin{equation}
G(t)=L\gamma!\,t^{-\gamma}+t^{-\delta}\eta(t) \qedhere
\end{equation}
\end{question}
%
%
\ques{1}{Here \verb@\ques{"your-num"}{}@ you have to enter the number of question}
%
%
\QUES{Above \verb@\QUES{}@ numbers the question using a predefined counter!}\\
\vspace{8pt}
\QUES{See this one is numbered one more than above question}\\ 
%
%
%
%\noindent\hrulefill\\
%==========================================================================
\begin{center}\deco{10pt}{44444444444}\end{center}
%==========================================================================
